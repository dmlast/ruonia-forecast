\section{Timelime}

\begin{center}
\small
\begin{tabular}{@{}p{2.9cm}p{1.3cm}p{9.5cm}@{}}
\toprule
\textbf{Дата} & \textbf{Этап} & \textbf{Ключевые deliverables} \\ \midrule
17 мая        & \textbf{W0}  & \checkmark\ сбор данных (RUONIA, макро, рынки)\\[-0.3ex]
              &             & \checkmark\ первичный EDA-ноутбук (статистика, графики) \\[0.6ex]

18–24 мая     & \textbf{W1} & Обзор литературы и Baseline: ARIMA, Васичек, EWMA; расчёт MAE/RMSE для $h=\{1,7,30\}$ \\[0.6ex]

25–31 мая     & \textbf{W2} & DeepAR \& TFT (1-шаговый forecast), валидация CRPS/Pinball, первый сравнительный график \\[0.6ex]

1–7 июня      & \textbf{W3} & Normalizing Flows (MAF/NSF) + Diffusion TimeGrad; multi-horizon density, Energy Score \\[0.6ex]

8–14 июня     & \textbf{W4} & Volatility block: GARCH-GJR, GAS, Bayesian SV; VaR$_{99}$/ES$_{97.5}$ back-тест, Купиц/Кристофферсен \\[0.6ex]

15–21 июня    & \textbf{W5} & Итоговые метрики + Diebold–Mariano, CPA-тесты; абляция признаков, SHAP-анализ TFT \\[0.6ex]

22–28 июня    & \textbf{W6} & Черновик отчёта, README для GitHub; слайды (10 стр.) для внутренней защиты \\[0.6ex]

29–30 июня    & \textbf{Buffer} & Полировка кода, оформление библиографии, загрузка финального PDF и ноутбуков \\ \midrule
\multicolumn{3}{@{}p{14.1cm}@{}}{\emph{Задача со звёздочкой (Neural SDE для совместного распределения)} планируется как расширение магистерского проекта: старт в июле 2025 после фиксации базовых результатов.} \\ \bottomrule
\end{tabular}
\end{center}

