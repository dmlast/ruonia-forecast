\section{Введение}

Цель проекта -- построить \emph{однофакторное} стохастическое
дифференциальное уравнение (СДУ), описывающее эволюцию рублевой
овернайт-ставки (RUONIA), с использованием метода \textbf{SINdy}
(\emph{Sparse Identification of Nonlinear Dynamics}).  
Однофакторная спецификация для краткосрочной процентной ставки
традиционно задаётся суммой функций \emph{дрейфа}
$\,\mu(r_t,t)\,$ и \emph{волатильности} $\,\sigma(r_t,t)\,$:
\begin{equation}
  dr_t = \mu(r_t,t)\,dt + \sigma(r_t,t)\,dW_t.
\end{equation}
Задача литературного обзора -- (i) проанализировать,
какие формы $\,\mu(\cdot)$ и $\,\sigma(\cdot)$ используются в современных
однофакторных моделях срочных ставок в мире,  
(ii) выделить рекомендации, применимые к RUONIA,  
(iii) обосновать выбор объясняющих переменных:
своповые котировки, доходности ОФЗ, курсы EUR/RUB, USD/RUB, CNY/RUB
и индекс Мосбиржи.

\section{Классические однофакторные СДУ для краткосрочных ставок}

\subsection{Модель Васичека}
Первой в литературу вошла гауссовская
модель \cite{Vasicek1977}:
$dr_t = \kappa(\theta - r_t)\,dt + \sigma\,dW_t$.
Линейный дрейф обеспечивает
экспоненциальную \emph{mean-reversion},
а константная волатильность -- аналитическую
удобность (закрытые формулы для ZCB и свопов).

\subsection{Модель Кокса-Инголлса-Росса (CIR)}
\cite{CIR1985} предложили квадрат-корневую
волатильность $\,\sigma\sqrt{r_t}\,$,
что гарантирует неотрицательность процентных ставок
и лучше описывает вариативность вблизи нуля.
Однако в периоды высоких ставок (как в России)
гетероскедастичность CIR может переоценивать дисперсию.

\subsection{Расширенный Васичек (Халл-Уайт)}
\cite{HullWhite1990} расширили модель Васичека,
разрешив $\,\theta(t)\,$ быть функцией времени.
Это позволяет точно калиброваться под текущую кривую
без добавления факторов.

\subsection{Логнормальная динамика (Black-Karasinski)}
В \cite{BlackKarasinski1991} дрейф остался
средне-обратимым, но логарифм ставки следует OU-процессу,
что исключает отрицательные значения при умеренной
комплексности.

\subsection{Сравнение для позиций овернайт}
Для высокочастотных (O/N) ставок классические
модели дополняют \emph{сезонными} и \emph{регуляторными}
компонентами -- см. \cite{Beltran2024},
где для SOFR к гауссовому дрейфу добавлены скачки,
ассоциированные с концом отчётных периодов.

\section{Современные спецификации для RFR-бенчмарков}

Окончательный отказ от LIBOR вызвал волну исследований
новых \textbf{overnight} бенчмарков (SOFR, \texteuro STR, SONIA).  
\cite{Fontana2022} формулируют общий
\emph{affine-семимартингал} с дискретными
дисконтиниутетами (policy jumps), а
\cite{Beltran2024} выявляют значимость пуассоновских
скачков в SOFR.  
Для прикладных расчётов используются упрощённые
однофакторные модели -- напр.\ однафакторный
Vasicek для совместной динамики SOFR и unsecured rate
\cite{Smith2025}.  
Сезонность (конец месяца, отчётный квартал)
обычно аппроксимируют синусоидальной добавкой
к дрейфу: $\mu(r_t,t)=\kappa(\theta-r_t)+a\sin(2\pi t/T)$,
что обеспечивает компактное описание периодических всплесков.

\section{Подход SINdy и недавние результаты}

Алгоритм SINdy \cite{BruntonKutz2022}
решает обратную задачу: по наблюдениям $\{r_t\}$ и
оценкам $\{\dot r_t\}$ выбирает редкий набор
функций из библиотеки $\Theta(r)$, минимизируя
$\|\dot r - \Theta \Xi\|_2$ при LASSO- или
SBL-регуляризации.  
Для финансовых рядов важна устойчивость к шуму и
редким экстремам; последние разработки --
robust SINDy \cite{Fung2025} и SINDyG
для сетевых данных \cite{SINDyG2024} --
дают пригодные процедуры отбора
со штрафом за мультиколлинеарность.

\section{Выбор функциональных форм для RUONIA}

Исходя из эмпирики российского рынка:

\begin{itemize}
      \item \textbf{Дрейф.} Linear mean-reversion к плавающему
        уровню $\theta(t)$ (аналог Hull-White) +
        сезонная синусоида на конец месяца (вид "tax day" для рубля).
        Поддерживается выводами ЦБ РФ о волатильности спрэда
        RUONIA-key rate \cite{BoRTrans2024}.

      \item \textbf{Волатильность.}
        На интервале ставок $15\!-\!25\%$ дисперсия близка к пропорциональной,
        что указывает на модель CIR; однако для отрицательных
        значений невозможна.  
        Компромисс -- \emph{shifted-CIR} с сдвигом $\gamma$
        или lognormal (BK) при низких ставках.
\end{itemize}


\section{Обоснование наблюдаемых переменных}

\begin{itemize}
  \item \textbf{Своповые цены}  Отражают безрисковую кривую
        и стабилизируют оценку $\theta(t)$
        для однофакторных моделей \cite{HullWhite1990}.
  \item \textbf{Доходности ОФЗ}  В отчётах Банка России
        OFZ-доходности фигурируют как основной
        high-frequency индикатор монетарных условий,
        коррелируя с RUONIA-spread \cite{BoRTrans2024}.
  \item \textbf{Курсы EUR, USD, CNY}  Реакция рублевой
        краткосрочной ставки на внешний шок
        выражена через FX-премию риска (covered interest-rate parity).
  \item \textbf{Индекс Мосбиржи}  Широкий индикатор
        ликвидности и ожиданий инфляции,
        влияющих на короткий конец кривой.
\end{itemize}

Эти переменные входят в библиотеку кандидатов
SINdy как потенциальные \emph{экзогенные} регрессоры.

\begin{table}[htbp]
\small                         
\centering
\caption{Библиотека кандидатных термов $\Theta(r_t,\mathbf{x}_t)$ для алгоритма \textsc{SINDy}}
\label{tab:sindy-library}
\renewcommand{\arraystretch}{1.15}
\begin{tabularx}{\textwidth}{@{}l>{\raggedright\arraybackslash}X>{\raggedright\arraybackslash}m{2.8cm}@{}}
\toprule
\textbf{Функция} & \textbf{Экономическая / статистическая мотивация} & \textbf{Литература} \\
\midrule
$1$                                   & Базовый уровень дрейфа (константный компонент)                                          & \cite{Vasicek1977,HullWhite1990} \\[2pt]
$r_t$                                 & Линейная mean-reversion (гауссовский Васичек)                                            & \cite{Vasicek1977} \\[2pt]
$r_t^{2}$                             & Квадратичный дрейф / 3/2-модель                                                         & \cite{Rogers2024threehalf,Chan1992} \\[2pt]
$r_t^{3}$                             & Дополнительная кривизна при высоких ставках                                              & \cite{Rogers2024threehalf} \\[2pt]
$\sqrt{r_t}$                          & Квадрат-корневой диффузионный термин (CIR)                                               & \cite{CIR1985} \\[2pt]
$\log r_t$                            & Логнормальная спецификация (Black-Karasinski)                                            & \cite{BlackKarasinski1991} \\[2pt]
$1/r_t$                               & Reciprocal-root-process -- подчёркнутая гетероскедастичность                               & \cite{Andersen2005} \\[2pt]
\midrule
$\sin(2\pi t/T)$                      & Сезонность (конец месяца/квартала)                                                      & \cite{Beltran2024,Jha2025} \\[2pt]
$\cos(2\pi t/T)$                      & Фазовый сдвиг той же сезонности                                                         & \cite{Beltran2024,Jha2025} \\[2pt]
$r_t\sin(2\pi t/T)$                   & Усиление сезонного эффекта при высоких ставках                                           & \cite{Beltran2024} \\[2pt]
$r_t\cos(2\pi t/T)$                   & То же -- с косинусоидой                                                                   & \cite{Beltran2024} \\[2pt]
\midrule
$\Delta y^{\text{OFZ}}_{t,k}$         & Шок кривой ОФЗ -- канал гос.долга                                                        & \cite{BoRTrans2024} \\[2pt]
$r_t\,\Delta y^{\text{OFZ}}_{t,k}$    & State-dependent монетарный эффект                                                       & \cite{BoRTrans2024} \\[2pt]
$\Delta s^{\text{swap}}_{t}$          & Сдвиг swap-rate (якорь безрисковой кривой)                                              & \cite{HullWhite1990} \\[4pt]
\midrule
\multicolumn{3}{@{}l}{\bfseries FX-шоки (USD, EUR, CNY)}\\[-2pt]
$\Delta\ln\text{FX}_{t}^{c}$                       & Девальвационный шок в валюте $c\!\in\!\{\mathrm{USD,EUR,CNY}\}$ (covered interest parity) & \cite{chen2023macrofinancial,huang2024transfer} \\[2pt]
$\lvert\Delta\ln\text{FX}_{t}^{c}\rvert$           & Амплитуда FX-колебаний -- прокси внешней волатильности                                   & \cite{chen2023macrofinancial} \\[2pt]
$\Delta\ln\text{FX}_{t-1}^{c}$                     & Лаг FX-шока -- инерционное влияние на денежный рынок                                     & \cite{chen2023macrofinancial} \\[2pt]
$r_t\,\Delta\ln\text{FX}_{t}^{c}$                  & Взаимодействие уровня ставки и внешнего шока                                            & \cite{chen2023macrofinancial,huang2024transfer} \\[4pt]
\midrule
$\Delta\ln\text{MOEX}_t$              & Шок ликвидности / инфляционные ожидания через рынок акций                               & \cite{BoRTrans2024} \\[2pt]
\midrule
$\hat h_{t-1}$                        & Условная дисперсия (GARCH/GAS) -- proxy внутренней волатильности                         & \cite{Engle1982,Bollerslev1986} \\[2pt]
$J^{\text{EoM}}_t$                    & 0–1 индикатор «последний рабочий день месяца» (прыжки ликвидности)                      & \cite{Beltran2024} \\[2pt]
\bottomrule
\end{tabularx}
\end{table}



\section{Заключение}
Современная практика моделирования
overnight-ставок сводится к углублённым,
но всё ещё \emph{однофакторным} СДУ,
обогащённым переменными сезонности
и скачков денежного рынка.  
Метод SINdy предоставляет системный способ
отобрать минимальный, интерпретируемый набор
термов, что позволяет гибко сочетать
линейный дрейф, квадрат-корневую
или логнормальную волатильность и экзогенные
финансовые факторы -- т.е. ровно то,
что требуется для RUONIA.