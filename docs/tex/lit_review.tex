\section{Введение}
Однофакторные стохастические модели ставки (short-rate models) остаются основным инструментом оценивания безрисковой кривой доходности и ценообразования процентных деривативов. После реформы эталонных ставок (замена LIBOR на \textsc{risk-free rates}~— SOFR, SONIA, \euro{}STR и др.) ключевым объектом моделирования стала именно овернайт-ставка. В российском контексте её аналогом является \textsc{ruonia}. Цель раздела~— проследить эволюцию однофакторных СДУ, показать обоснование форм дрейфа и волатильности в современной литературе и аргументировать выбор рыночных факторов (свопы, ОФЗ, валюты, индекс Мосбиржи), которые мы будем использовать при калибровке модели RUONIA.

\section{Классические однофакторные модели}
\subsection{Васичек (1977)}
Модель Васичека описывает ставку $r_t$ процессом Орнстейна–Уленбека \autocite{Vasicek1977}:
\begin{equation}
  \mathrm{d}r_t = a\bigl(b - r_t\bigr)\,\mathrm{d}t + \sigma\,\mathrm{d}W_t.
\end{equation}
Линейный дрейф обеспечивает средневозвратность, постоянная волатильность $\sigma$ порождает нормальное распределение, допускающее отрицательные значения — что оказалось актуальным в эпоху нулевых ставок.

\subsection{Cox--Ingersoll--Ross (1985)}
Чтобы исключить отрицательные значения, CIR-модель вводит зависимость волатильности от уровня ставки \autocite{CIR1985}:
\begin{equation}
  \mathrm{d}r_t = a\bigl(b - r_t\bigr)\,\mathrm{d}t + \sigma\sqrt{r_t}\,\mathrm{d}W_t.
\end{equation}
Эмпирические тесты Chan \textit{et al.} \autocite{Chan1992} показали эффект уровня: дисперсия растет при повышении $r_t$. Значения степени $\gamma\approx1.3\text{--}1.6$ в расширенной формуле $\sigma\,r_t^{\gamma}$ подтверждены и на рынках евро и фунта.

\subsection{Hull--White (1990) и Black--Karasinski (1991)}
Hull--White сохраняет гауссовский характер, но позволяет параметрам быть функцией времени, что дает точную калибровку к начальной кривой \autocite{HullWhite1990}. Black--Karasinski моделирует логарифм ставки, обеспечивая строго положительные значения и логнормальное распределение \autocite{BlackKarasinski1991}.

\section{Современные расширения}
\subsection{Циклические компоненты}
Jha \autocite{Jha2025} предложил синусоидальную модификацию Hull--White, где скорость реверсии колеблется:
\[
  a(t) = a_0 + \Delta a \sin(\omega t).
\]
Это улучшает описание 20-летних макроциклов, выявленных в ставках SOFR.

\subsection{Сезонность овернайт-ставок}
ФРБ Нью-Йорка \autocite{FRBNY2020} документировал квартальные всплески SOFR, связанные с балансировочными ограничениями банков. Практика прогнозирования включает гармонические члены $S\sin(2\pi t) + C\cos(2\pi t)$ в дрейфе; аналогичные «налоговые» эффекты отмечены и для RUONIA.

\subsection{Отрицательные ставки и нелинейные диффузии}
Распространение отрицательных ставок стимулировало модели shifted-CIR и нормальные модели (Hull--White) с волатильностями Башелье. Для нивелирования нулевой границы предлагается модель $3/2$ ($\sigma\,r_t^{3/2}$), сохраняющая высокую дисперсию при малых $r_t$ \autocite{Rogers2024threehalf}.

\section{Реформа эталонных ставок}
Переход на RFR (SOFR, SONIA, \euro{}STR, TONA) означал, что базовой случайной величиной становится именно овернайт-безрисковая ставка \autocite{BIS2019}. В отличие от LIBOR, она более волатильна внутримесячно, что усиливает требования к модели волатильности и сезонных поправок.

\section{Макро- и рыночные факторы, влияющие на ставки}
\begin{itemize}
  \item \textbf{Доходности ОФЗ и свопы.} Формируют ожидание долгосрочного уровня $b$; Банк Канады \autocite{BoC2007} и Банк России \autocite{BoCRu2023} используют своповую кривую в качестве ориентира.
  \item \textbf{Валютный курс.} Процентный паритет связывает дифференциал ставок с форвардной премией; МВФ \autocite{IMF2023} фиксирует сильную реакцию курсов EM-валют на изменения местных ставок.
  \item \textbf{Фондовый рынок.} Низкие ставки стимулируют рост акций, высокие — охлаждают рынок \autocite{RBC2024}. Индекс Мосбиржи служит индикатором финансовых условий.
\end{itemize}

\section{Импликации для модели RUONIA}
На основании обзора целесообразно использовать спецификацию
\[
  \mathrm{d}r_t = \bigl[a_0 + S\sin(2\pi t) + C\cos(2\pi t) - a\,r_t\bigr]\mathrm{d}t
    + \sigma\,r_t^{\gamma}\,\mathrm{d}W_t,
  \quad \gamma\in[0.5,1.5],
\]
где $a_0,S,C$ калибруются по своповой кривой и доходностям ОФЗ, а показатель $\gamma$ — из регрессии дисперсии на уровень RUONIA. Классификация режимов по валютным курсам и индексу Мосбиржи позволит дополнительно проверять стабильность параметров.
