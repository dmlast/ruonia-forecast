\section*{Приложение 1. Метаданные публикации и применения данных}
\addcontentsline{toc}{section}{Приложение 1. Метаданные публикации и применения данных}

\begin{table}[htbp]
\centering
\scriptsize
\setlength{\tabcolsep}{2pt}
\renewcommand{\arraystretch}{1.15}
\caption{Время публикации и лаг применения ключевых источников данных}
\label{tab:data-metadata}

\begin{tabularx}{\linewidth}{@{}
  >{\raggedright\arraybackslash}X        % Показатель
  >{\centering\arraybackslash}p{1.9cm}   % Публикация
  >{\centering\arraybackslash}p{0.9cm}   % Лаг
  >{\centering\arraybackslash}p{1.5cm}   % Применение
  >{\raggedright\arraybackslash}X@{} }    % Календарь
\toprule
\multicolumn{1}{c}{\textbf{Показатель}} &
\multicolumn{1}{c}{\textbf{Публикация}} &
\multicolumn{1}{c}{\textbf{Лаг}} &
\multicolumn{1}{c}{\textbf{Применение}} &
\multicolumn{1}{c}{\textbf{Календарь}} \\
\midrule
RUONIA (overnight) & 
  \textasciitilde 18:30\,(UTC+3)\footnotemark[1] & 
  +1  & DATE+1 & RUONIA (NFA)\footnotemark[2] \\

OIS-фиксы (swap curve) & 
  \textasciitilde 19:00\,(UTC+3)\footnotemark[1] & 
  0   & DATE   & RUONIA (NFA)\footnotemark[2] \\

Курсы ЦБ РФ & 
  \textasciitilde 15:30\,(UTC+3)\footnotemark[3] & 
  +1  & DATE+1 & Банковские дни ЦБ РФ\footnotemark[4] \\

IMOEX close & 
  \textasciitilde 18:50\,(UTC+3)\footnotemark[5] & 
  0   & DATE   & Биржевой календарь MOEX\footnotemark[6] \\

OFZ zero-curve yields & 
  \textasciitilde 19:15\,(UTC+3)\footnotemark[5] & 
  0   & DATE   & Биржевой календарь MOEX\footnotemark[6] \\
\bottomrule
\end{tabularx}

\vspace{0.6em}
\footnotesize
\textbf{Примечания.} Символ `\textasciitilde` обозначает приблизительное время публикации (UTC+3).  
Лаг «+1» означает, что значение, опубликованное в T, становится применимым с начала дня T\,+1.

\footnotetext[1]{Регламент публикации ставок RUONIA и ROISfix \cite{nfa_reglament}.}
\footnotetext[2]{Календарь публикаций RUONIA \cite{nfa_calendar}.}
\footnotetext[3]{Информационное письмо Банка России о курсах валют \cite{cbr_fx_letter}.}
\footnotetext[4]{Календарь банковских выходных ЦБ РФ \cite{cbr_calendar}.}
\footnotetext[5]{Post-trade отчёты MOEX \cite{moex_post_trade}.}
\footnotetext[6]{Торговый календарь MOEX \cite{moex_calendar}.}
\end{table}






