\section{Введение}
\subsection*{Зачем этот раздел}     
Сначала мы представим краткие определения ключевых ставок и метрик для более прозрачной методологии. 
Это позволит уточнить, какие именно величины мы намерены прогнозировать 
и какие критерии точности будут использованы. 
Вслед за этим будет изложена мотивация прогнозирования
\textsc{ruonia} и возможный спектр методов.

\subsection{Словарь ключевых понятий}

\textbf{RUONIA}\footnote{Russian Overnight Index Average. Методика — \autocite{cbr_method2024}.} —
средневзвешенная ставка по необеспеченным межбанковским займам на один
день (\emph{overnight}). Банк России собирает сделки, считает
взвешенное среднее и публикует его на следующий рабочий день. Это
единственный публичный индикатор того, «сколько сейчас действительно стоит рубль
на межбанковском рынке».

\textbf{Овернайт-ставки ARR}:  
SOFR (США), SONIA (Британия), €STR (Евро) появились после
отказа от LIBOR в 2021г. \autocite{arrc2021}. Их объединяет ежедневная
публикация и использование как базовой ставки по деривативам.

\textbf{Базис-пойнт (bp)} — одна сотая процента, $1\,\text{bp}=0.01\%$.

\textbf{Волатильность} —
\[
\sigma_t^{2}= \operatorname{Var}\bigl(\Delta r_{t+1}\mid\mathcal F_t\bigr)
\]

где $\mathcal F_t$ — «всё, что мы знаем сегодня»: прошлые значения
RUONIA, ключевой ставки, курс USD/RUB, индексы ликвидности и т.д. 
Может браться также стандартное отклонение или логарифм стандартного отклонения.

\textbf{VaR\,$_{99}$} — такое $q$, что однодневный убыток портфеля
превысит $q$ не чаще 1 \% случаев. Регулятор требует бэктест
на последних 250 дней и штрафует, если нарушений слишком много
\autocite{basel2023}.

\subsection{Почему прогноз RUONIA интересен для ML}

Точность прогноза overnight-ставки на $\pm5$ bp уже экономит сотни тысяч долларов в день
на портфеле \SI{1e9}{RUB} O/N-свопов
\autocite{duffie2023onn}.  В России для прогнозирования RUONIA традиционно применяют сочетание:

\begin{itemize}
  \item \textbf{Модель Васичека}:  
    однопараметрическое СДУ
    \[
      \mathrm{d}r_t = \kappa(\theta - r_t)\,\mathrm{d}t + \sigma\,\mathrm{d}W_t,
    \]
    где 
    \(\kappa\) — скорость возврата к среднему \(\theta\), 
    \(\sigma\) — амплитуда случайных флуктуаций, 
    \(W_t\) — винеровский процесс.  
    Оно даёт явную формулу для условного среднего
    \(\displaystyle \mathbb{E}[r_{t+h}\mid \mathcal F_t] = \theta + (r_t-\theta)e^{-\kappa h}\),
    но не оценивает распределение и реагирует на шоки плавно.
  \item \textbf{EWMA-волатильность}:  
    экспоненциально взвешенная скользящая оценка дисперсии
    \[
      \sigma_t^2 = \lambda\,\sigma_{t-1}^2 + (1-\lambda)\,\bigl(r_t - r_{t-1}\bigr)^2,
      \quad \lambda\in(0,1).
    \]
    В стандарте Basel II обычно берут \(\lambda=0.94\).  
    Метод прост в реализации, но «запаздывает» при резких изменениях:
    крупный скачок волатильности долго «расплывается» по окну.
\end{itemize}

Эти подходы дают только точечный прогноз среднего и сглаженную
волатильность, но слабо реагируют на внезапные изменения колебаний, к примеру,
после неожиданных заявлений Центробанка, что снижает точность оценки
рисков.  


\subsection{Актуальные результаты и мотивация}

За последние два года вышла серия статей, показывающих, что
\emph{вероятностные} и \emph{генеративные} модели задают новый
бенчмарк для прогнозов o/n-ставок (SOFR, €STR, SONIA):

\begin{itemize}
  \item \textbf{DeepAR / DeepState}.  Выигрыш до 25 \% по CRPS и
        Pinball-loss против ARIMA для SOFR \autocite{salinas2020deepar}.
  \item \textbf{TFT + Quantile Loss}.  Интервалы 5–95 \%, стресс-сценарии
         \autocite{lim2024tft,vaswani2024transformers}.
  \item \textbf{Normalizing Flows \& Diffusion Models}.  Улучшение
        VaR~(99 \%) на 30 \% относительно GARCH
        \autocite{rasouli2023flows,zangeneh2024diffusion,xu2024deep}.
  \item \textbf{Bayesian non-parametrics}.  Гибкая аппроксимация
        условной плотности без заданного семейства
        \autocite{alvarez2024bayesian}.
  \item \textbf{Quantile Regression Forests \& GNN}.  Дешёвые в
        обучении распределённые прогнозы и учёт сетевой структуры
        рынков \autocite{liu2024robust,hayashi2024graphnets}.
  \item \textbf{Large-Language / Multimodal models}.  Учёт новостей
        ФРС и текстовых факторов снижает ES-backtest
        \autocite{gao2024fedspeak,silva2024multimodal,chang2024forecasting}.
\end{itemize}

Для \textsc{ruonia} подобного анализа нет.  
Мы хотим построить \emph{полное} распределение
\(P(r_{t+h}\mid\mathcal F_t)\) и проверить, насколько современные
глубокие модели (TFT, DeepAR, NF, Diffusion, GNN) превосходят
базовый подход «Васичек + EWMA».  
Долгосрочная цель — Neural SDE / Neural PDE: обучить
дифференциальное уравнение, объясняющее «длинную память» RUONIA.

\section{Research Questions and Objectives}

\begin{enumerate}
  \item \textbf{RQ1}: Улучшают ли TFT, DeepAR, Normalizing Flows и
        Diffusion-TimeGrad качество \emph{полного} прогноза
        (CRPS, Energy, Pinball) на горизонтах
        \(h \in \{1,7,30,180,365\}\)
        по сравнению с базовой связкой «Васичек + EWMA»?
  \item \textbf{RQ2}: Снижают ли продвинутые модели волатильности
        (GARCH-GJR, GAS, EGARCH, Bayesian SV) частоту нарушений
        99 \% VaR и Expected Shortfall
        относительно фильтра EWMA (\(\lambda=0.94\))?
  \item \textbf{RQ3}\emph{(*)} : Может ли \emph{совместное} генеративное решение
        — Conditional NF или Neural SDE — предоставить согласованную
        многогоризонтную плотность \(\{r_{t+h}\}_{h}\) и тем самым
        улучшить back-тест ES? \footnote{ 
        Эта задача рассматривается как «задача со звёздочкой»:
        её планируется детально развернуть уже в рамках магистерского
        тезиса, после отработки основной части моделирования.}
\end{enumerate}

Наша цель — получить устойчивое распределение
\(P(r_{t+h}\mid\mathcal F_t)\) для
\(h \in \{1,7,30,180,365\}\) и тем самым закрыть пробел между
«точечным» прогнозом ставки и реальными требованиями риск-менеджмента.


\section{Methodology}

\subsection{Probabilistic Sequence Models}

\begin{itemize}
  \item \textbf{Baseline (ARIMA + Vasicek + EWMA).}  ARIMA$(p,d,q)$
        обеспечивает минимально-адекватную авто-коррекционную структуру;
        Васичек даёт экономически интерпретируемый
        параметр $\kappa$; EWMA($\lambda{=}0.94$) – промышленный
        стандарт Банка России, поэтому служит референсом.
  \item \textbf{DeepAR / DeepState.} (Student-$t$, NB) обучается end-to-end на
        \emph{условной} плотности $\!P(r_{t+1}\mid\mathcal F_t)$,
        хорошо работает на нестационарных рядовых потоках
        \autocite{salinas2020deepar}.
  \item \textbf{Temporal Fusion Transformer (TFT).}  Комбинирует
        attention + гating; quantile-loss $L_\tau$ сразу выдаёт набор
        доверительных интервалов, а MC-dropout калибрует 
        неопределённость \autocite{lim2024tft,vaswani2024transformers}.
  \item \textbf{Normalizing Flows (MAF / NSF).}  Обратимо
        трансформируют сложную плотность в $\mathcal N(0,1)$,
        позволяя моделировать асимметрию и тяжёлые хвосты
        \autocite{rasouli2023flows}.
  \item \textbf{Diffusion (TimeGrad).}  Обучаем «шум $\rightarrow$ ряд»,
        получаем мульти-шаговые сценарии без ошибки накопления
        \autocite{zangeneh2024diffusion}.
  \item \textbf{LLM + Time-mix.}  ФРС, макро-факторы и
        исторический ряд объединяются в единую эмбеддинг-матрицу,
        что улучшает tails / ES прогноза
        \autocite{gao2024fedspeak,chang2024forecasting}.
\end{itemize}

\subsection{Volatility \& Density Models}

\begin{itemize}
  \item \textbf{GARCH / EGARCH / GJR.}  Захватывают
        кластеризацию волатильности; сравниваем Normal и $t$-шоки.
  \item \textbf{GAS (Generalised AR Score).}  Обновляет
        $\sigma_{t}^{2}$ градиентом log-likelihood, поэтому быстрее реагирует
        на рыночные шоки \autocite{creal2013gas}.
  \item \textbf{Bayesian SV.}  MCMC NUTS даёт постериор на
        $\sigma_t$, что нужно для full-Bayes ES / VaR
        \autocite{alvarez2024bayesian}.
\end{itemize}

\subsection{Evaluation}

\begin{itemize}
  \item \textit{Point}: \textbf{MAE, RMSE} – привычные банковские KPI
        для одной цифры.
  \item \textit{Distribution}: \textbf{CRPS} (калибровка всей CDF),
        \textbf{Pinball Loss} (проверка квантилей 5 \% / 95 \%),
        \textbf{Energy Score} – многомерная общая мера для сценариев.
  \item \textit{Risk}: \textbf{VaR$_{99}$, ES$_{97.5}$} +
        Купиц, Кристофферсен – регуляторные back-тесты (Базель IV).
  \item \textit{Forecast comparison}: \textbf{Diebold–Mariano (QLIKE)}
        и \textbf{Giacomini–White CPA} – формальная проверка,
        «статистически ли» новая модель лучше Васичека.
\end{itemize}
